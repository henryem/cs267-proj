\documentclass{article}
\usepackage{amsmath, amssymb}
\usepackage[retainorgcmds]{IEEEtrantools}
\usepackage{algorithm}
\usepackage{algpseudocode}
\usepackage{filecontents}
\usepackage{hyperref}
\author{Derek Kuo and Henry Milner}
\title{CS267 Project Proposal: Algorithmic Lovasz Local Lemma}
\date{4/6/15}

% Some convenience functions for homework problems.
\newcommand{\problem}[1]%
  {\section*{#1.}}

\newcommand{\problemSubpart}[1]%
  {\noindent\emph{#1.}}

\newcommand{\problemNamedSubpart}[1]%
  {\noindent\emph{#1}}

% Some convenience functions for note-taking.
\newcommand{\topic}[1]%
  {\section*{#1}}

% Some functions for general use.

\def\seqn#1\eeqn{\begin{align}#1\end{align}}

\newcommand{\vecName}[1]%
  {\boldsymbol{#1}}

\newcommand{\io}%
  {\text{ i.o. }}

\newcommand{\eventually}%
  {\text{ eventually }}

\newcommand{\tr}%
  {\text{tr}}

\newcommand{\Cov}%
  {\text{Cov}}

\newcommand{\adj}%
  {\text{adj}}

\newcommand{\funcName}[1]%
  {\text{#1}}

\newcommand{\hasDist}%
  {\sim}

\DeclareMathOperator*{\E}%
  {\mathbb{E}}

\newcommand{\Var}%
  {\text{Var}}

\newcommand{\std}%
  {\text{std}}

\newcommand{\grad}%
  {\nabla}

\DeclareMathOperator*{\argmin}{arg\,min}

\DeclareMathOperator*{\argmax}{arg\,max}

\newcommand{\inprod}[2]%
  {\langle #1, #2 \rangle}

\newcommand{\dd}[1]%
  {\frac{\delta}{\delta#1}}

\newcommand{\Reals}%
  {\mathbb{R}}

\newcommand{\indep}%
  {\protect\mathpalette{\protect\independenT}{\perp}} \def\independenT#1#2{\mathrel{\rlap{$#1#2$}\mkern2mu{#1#2}}}

\newcommand{\defeq}%
  {\buildrel\triangle\over =}

\newcommand{\defn}[1]%
  {\emph{Definition: #1}\\}

\newcommand{\example}[1]%
  {\emph{Example: #1}\\}

\newcommand{\figref}[1]%
  {\figurename~\ref{#1}}

\newtheorem{theorem}{Theorem}[section]
\newtheorem{lemma}[theorem]{Lemma}
\newenvironment{proof}[1][Proof]{\begin{trivlist}
\item[\hskip \labelsep {\bfseries #1}]}{\end{trivlist}}

\begin{filecontents}{\jobname.bib}
@article{erdos1975problems,
  title={Problems and results on 3-chromatic hypergraphs and some related questions},
  author={Erdos, Paul and Lov{\'a}sz, L{\'a}szl{\'o}},
  journal={Infinite and finite sets},
  volume={10},
  number={2},
  pages={609--627},
  year={1975}
}
@article{beck1991algorithmic,
  title={An algorithmic approach to the Lov{\'a}sz local lemma. I},
  author={Beck, J{\'o}zsef},
  journal={Random Structures \& Algorithms},
  volume={2},
  number={4},
  pages={343--365},
  year={1991},
  publisher={Wiley Online Library}
}
@article{moser2010constructive,
 author = {Moser, Robin A. and Tardos, G\'{a}bor},
 title = {A Constructive Proof of the General Lov\ÁSz Local Lemma},
 journal = {J. ACM},
 issue_date = {January 2010},
 volume = {57},
 number = {2},
 month = feb,
 year = {2010},
 issn = {0004-5411},
 pages = {11:1--11:15},
 articleno = {11},
 numpages = {15},
 url = {http://doi.acm.org/10.1145/1667053.1667060},
 doi = {10.1145/1667053.1667060},
 acmid = {1667060},
 publisher = {ACM},
 address = {New York, NY, USA},
 keywords = {Constructive proof, Lov\'{a}sz local lemma, parallelization},
} 
@article{haeupler2011new,
  title={New constructive aspects of the lovasz local lemma},
  author={Haeupler, Bernhard and Saha, Barna and Srinivasan, Aravind},
  journal={Journal of the ACM (JACM)},
  volume={58},
  number={6},
  pages={28},
  year={2011},
  publisher={ACM}
}
@inproceedings{chung2014distributed,
  title={Distributed algorithms for the Lov{\'a}sz local lemma and graph coloring},
  author={Chung, Kai-Min and Pettie, Seth and Su, Hsin-Hao},
  booktitle={Proceedings of the 2014 ACM symposium on Principles of distributed computing},
  pages={134--143},
  year={2014},
  organization={ACM}
}
@inproceedings{freer2010probabilistic,
  title={When are probabilistic programs probably computationally tractable?},
  author={Freer, Cameron E and Mansinghka, Vikash K and Roy, Daniel M},
  booktitle={NIPS Workshop on Advanced Monte Carlo Methods with Applications},
  year={2010}
}
@article{wainwright2008graphical,
  title={Graphical models, exponential families, and variational inference},
  author={Wainwright, Martin J and Jordan, Michael I},
  journal={Foundations and Trends{\textregistered} in Machine Learning},
  volume={1},
  number={1-2},
  pages={1--305},
  year={2008},
  publisher={Now Publishers Inc.}
}
@inproceedings{papadimitriou1991selecting,
  title={On selecting a satisfying truth assignment},
  author={Papadimitriou, Christos H},
  booktitle={Foundations of Computer Science, 1991. Proceedings., 32nd Annual Symposium on},
  pages={163--169},
  year={1991},
  organization={IEEE}
}
@inproceedings{steurer2010fast,
  title={Fast SDP algorithms for constraint satisfaction problems},
  author={Steurer, David},
  booktitle={Proceedings of the twenty-first annual ACM-SIAM symposium on Discrete Algorithms},
  pages={684--697},
  year={2010},
  organization={Society for Industrial and Applied Mathematics}
}
@inproceedings{polik2007sedumi,
  title={SeDuMi: a package for conic optimization},
  author={Polik, Imre and Terlaky, Tamas and Zinchenko, Yuriy},
  booktitle={IMA workshop on Optimization and Control, Univ. Minnesota, Minneapolis},
  year={2007}
}
@article{malaguti2010survey,
  title={A survey on vertex coloring problems},
  author={Malaguti, Enrico and Toth, Paolo},
  journal={International Transactions in Operational Research},
  volume={17},
  number={1},
  pages={1--34},
  year={2010},
  publisher={Wiley Online Library}
}
@article{gomes2008satisfiability,
  title={Satisfiability solvers},
  author={Gomes, Carla P and Kautz, Henry and Sabharwal, Ashish and Selman, Bart},
  journal={Foundations of Artificial Intelligence},
  volume={3},
  pages={89--134},
  year={2008},
  publisher={Elsevier}
}
@article{balint2013generating,
  title={Generating the uniform random benchmarks for SAT Competition 2013},
  author={Balint, Adrian and Belov, Anton and Heule, Marijn JH and J{\"a}rvisalo, Matti},
  journal={Proceedings of SAT Competition},
  pages={97--98},
  year={2013}
}
@article{belov2014application,
  title={The Application and the Hard Combinatorial Benchmarks in SAT Competition 2014},
  author={Belov, Anton and Heule, Marijn JH and Diepold, Daniel and J{\"a}rvisalo, Matti},
  journal={SAT COMPETITION 2014},
  pages={81}
}
\end{filecontents}
\immediate\write18{bibtex \jobname}

\begin{document}
\maketitle

The algorithmic Lovasz Local Lemma (LLL) is a recent-developed class of randomized algorithms for solving certain easy instances of generically-hard combinatorial problems.  The seminal paper in the field is by Moser and Tardos \cite{moser2010constructive} in 2010.  For many problems, LLL algorithms theoretically lend themselves well to parallelism, as demonstrated in \cite{moser2010constructive} and expanded upon in more recent work \cite{chung2014distributed,haeupler2011new}.  However, we are aware of no published empirical tests of the efficiency of LLL algorithms in practice.  In our project, we will implement parallel LLL algorithms for graph coloring and k-SAT, and we will compare their performance on benchmarks against other algorithms.

Since the reader may be unfamiliar with this work, we will provide a brief introduction before outlining our project goals.

\section{A brief introduction to LLL algorithms}
The basic setup for LLL algorithms is as follows: Let $\mathcal{A} = \{A_1, \cdots, A_m\} \in \{0,1\}^m$ be a set of discrete variables (which we take to be binary for sake of exposition), and let $\mathcal{E} = \{E_1, \cdots, E_n\}$ be a set of functions $\{0,1\}^m \to \{0,1\}$ mapping the variables to truth values.  The $E_i$ are called ``events,'' and $E_i$ is said to ``happen'' under some assignment to $\mathcal{A}$ if $E_i(\mathcal{A}) == 1$.  We would like to find an assignment of the variables so that none of the events happen.

This framework encompasses many constraint satisfaction problems.  For example, in k-SAT, the $A_i$ are the problem variables, and each $E_i$ corresponds to a clause involving $k$ variables.  An event ``happens'' if the corresponding clause is false, and solving a k-SAT instance requires that all clauses are true, or equivalently that none of the events happen.  In k-SAT, each event $E_i$ is a function of only a smaller subset of the variables.  We denote that subset of $\mathcal{A}$ by $\operatorname{vbl}(E_i)$.  LLL algorithms typically need $|\operatorname{vbl}(E_i)| << |\mathcal{A}|$.

The simplest LLL algorithm is \emph{extremely} simple.  It is Algorithm \ref{alg:simple-lll} below.

\begin{algorithm}[H]
\label{alg:simple-lll}
\begin{algorithmic}
\State Initialize all the $A_i$ as IID samples from a Bernoulli($1/2$) distribution.
\While{Any of the $E_j$ happen under the current value of $\mathcal{A}$}
  \State Let $E_j$ be an arbitrary event that happens under $\mathcal{A}$.
  \State Reample each $A_i \in \operatorname{vlb}(E_j)$ IID Bernoulli($1/2$).
\EndWhile
\end{algorithmic}
\caption{The simplest LLL algorithm.}
\end{algorithm}

Of course, the decision version of k-SAT is NP-hard, so we would not expect this algorithm to work for all instances.  Moser and Tardos \cite{moser2010constructive} prove that it runs in expected polynomial time when the events are not individually too hard to satisfy with random assignment and when there are not too many dependencies among the events.  Concretely:
\begin{enumerate}
  \item Let $p$ be the maximum probability that any single event happens under a random assignment to all the variables.  (In k-SAT, this is $2^{-k}$, since a clause is unsatisfied only if all of its k variables are assigned incorrectly, and the single variable assignments happen independently with probability $1/2$.)
  \item Construct an undirected dependency graph for the events, in which there is an edge between $E_j$ and $E_k$ if they are statistically dependent under random assignment to the $A_i$.  In k-SAT, there is an edge between clauses $E_j$ and $E_k$ if they share any variable.  Let $d$ be the maximum degree of events in this graph.
\end{enumerate}
Moser and Tardos prove that the above algorithm finishes in expected polynomial time if $e p d < 1$, where $e$ is the base of the natural logarithm.

For example, a k-SAT problem satisfies this condition if each clause shares variables with no more than $\frac{2^k}{e}$ other clauses.  If we generate a random k-SAT instance with $m$ variables and $n$ clauses, the probability that two clauses share a variable is approximately $\frac{k^2}{m}$, so the expected value of $d$ is approximately $\frac{n k^2}{m}$.  In that case the proof by Moser and Tardos applies if $n$ is less than (approximately) $\frac{2^k m}{k^2 e}$.  For comparison, the maximum number of clauses in a k-SAT instance is $2^k {m \choose k}$.  It is known that random 3-SAT instances are hardest when $n \approxeq 4.26 m$ \cite{gomes2008satisfiability}, so these are relatively easy instances.

The event dependency graph is useful for seeing an easy way to parallelize the simple LLL algorithm.  Notice that if independent events $E_j$ and $E_k$ happen under the current assignment to $\mathcal{A}$, and event $E_j$'s variables are chosen for resampling, then (by independence) $E_k$ will still happen after the resampling, so the next iteration could resample its variables.  This means we can resample variables depended-on by any independent set of events that currently happen, and this resampling can be done in parallel.  The difficult part is finding a large independent set and communicating variable assignments.  Chung et al \cite{chung2014distributed} and Srinivasan et al \cite{haeupler2011new} give alternative parallel LLL algorithms (similar to Algorithm \ref{alg:simple-lll}, but somewhat more complicated) that can be run more efficiently on distributed computers.

\section{Project goals}
We list our goals in order of anticipated completion.  The goals marked (*) are stretch goals.

\begin{description}
  \item[1.] Implement the sequential version of Moser's algorithm for k-SAT and/or graph coloring.
  \item[2.] Implement the simple parallel version of Moser's algorithm for k-SAT and/or graph coloring in OpenMP or Spark.
  \item[3.] Implement Chung's distributed LLL algorithm for k-SAT and/or graph coloring in MPI or Spark.
  \item[4.] Synthesize a k-SAT benchmark that meets the theoretical requirements for polynomial running time of our distributed LLL algorithm.
  \item[*5.] Find more widely-used benchmarks for k-SAT and graph coloring (or possibly other related CSPs).  For k-SAT, we will use a subset of the benchmarks in SAT Competition 2014 \cite{balint2013generating, belov2014application}.  For graph coloring, we will be guided by a recent comprehensive survey of benchmarks and sovlers by Malaguti and Toth \cite{malaguti2010survey}.  We will focus on real-world problem instances, since LLL algorithms are unlikely to compete with other algorithms on artificial instances that are designed to be hard.
  \item[6.] Investigate the performance and scaling properties of our algorithms on these benchmarks.
  \item[*7.] Implement an existing k-SAT solver, or (preferably) use an off-the-shelf implementation.  Gomes et al \cite{gomes2008satisfiability} have a recent survey of SAT solvers.  Then we can compare single-node or parallel performance with the LLL-based solvers.
\end{description}

\bibliographystyle{plain}
\bibliography{\jobname}

\end{document}